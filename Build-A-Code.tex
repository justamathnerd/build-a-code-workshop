\documentclass{beamer}
\usetheme{metropolis}
%\usecolortheme{dove}
\usefonttheme{serif}


\title{Build-A-Code}
\subtitle{}
\date{May 5, 2017}
\author{Peter Keep}
\institute{Moraine Valley Community College}
\begin{document}

\maketitle
\section{The Problem}
\begin{frame}{The Problem}
If I start flipping a coin, which sequence of flips is more likely to occur first: HHT or HTT?\pause
\begin{itemize}
  \item Heads and Tails are equally likely \pause
  \item The probability of flipping HH and HT is equally likely \pause
  \item The wording of the question is important
\end{itemize}

\end{frame}

\section{Test It!}

\begin{frame}{Test It!}
Flip your coin, and write down the results of each flip in order. Stop when you see one of these patterns:

\textbf{Pattern 1:} HHT\\

\textbf{Pattern 2:} HTT\\\pause

\vspace{0.5cm}
\textit{BONUS: Count the number of coin flips it takes to see either pattern.}
\end{frame}

\plain{Which pattern comes up more often?\\
How many flips did it take?}

\section{Build a Program}


\plain{What do we need the program to do?}

\begin{frame}{Flip a Coin}
\texttt{import random}\\
\texttt{population=[0,1]} (0=heads, 1=tails)\\
\texttt{flip=random.choice(population)}\\


\end{frame}

\begin{frame}{Record Your Flip}
We'll set up a list to write down our coin flips (\texttt{fliplist})
% and a counter to record the number of flips (\texttt{i})
.\\ \pause

\vspace{0.5cm}
This will be our initial setup BEFORE we flip.\\
\
\texttt{fliplist=[]}\\
% \texttt{i=0}\\
\pause

\vspace{0.5cm}
This is what we'll do AFTER we flip.\\
\texttt{fliplist.append(flip)}\\
% \texttt{i=i+1}\\
\texttt{fliplist=fliplist[-3:]} (just look at the last 3 flips)
\end{frame}

\begin{frame}{Check the Patterns}
Initially, we'll define our patterns (\texttt{pattern1} and \texttt{pattern2}) as well as counters for when pattern 1 or pattern 2 wins (\texttt{p1wins} and \texttt{p2wins}).

% We'll also keep track of the ``winning'' lengths (\texttt{p1length} and \texttt{p2length}).\\

\vspace{0.5cm}\pause

\texttt{pattern1=[0,0,1]} \\
\texttt{pattern2=[0,1,1]}\\
\texttt{p1wins=0}\\
\texttt{p2wins=0}\\
% \texttt{p1length=[]}\\
% \texttt{p2length=[]}

\end{frame}

\begin{frame}{Check the Patterns}
We'll use ``if-then'' statements to add wins and lengths to the appropriate counters when either pattern shows up.\\
\vspace{0.5cm}\pause

\texttt{if fliplist==pattern1:}\\
\hspace{0.5cm}\texttt{p1wins=p1wins+1}\\
% \hspace{0.5cm}\texttt{p1length.append(i)}\\

\vspace{0.5cm}

\texttt{if fliplist==pattern2:}\\
\hspace{0.5cm}\texttt{p2wins=p2wins+1}\\
% \hspace{0.5cm}\texttt{p2length.append(i)}
\end{frame}

\begin{frame}{Keep Trying Until They Match}
This is the most technical part. We're going to put most of the pieces we've build inside something called a ``while loop.'' Since we'll stop flipping whenever we see either pattern, we'll keep flipping coins whenever our \texttt{fliplist} doesn't match \texttt{pattern1} or \texttt{pattern2}.\pause

\texttt{while fliplist != pattern1 and fliplist!=pattern2: }
\end{frame}

\plain{Bonus: Count the Flips!}

\begin{frame}{Bonus: Count the Flips!}
Each time we flip a coin, we should add a small counter. When we find the pattern, the counter should stop. We should store that number somewhere, and then when we're done, average the numbers for each pattern.\\ \pause
\vspace{0.2cm}

Setup: \texttt{p1length=[]}, \texttt{p2length=[]}, \texttt{i=0}\\ \pause
\vspace{0.2cm}

After each flip: \texttt{i=i+1}\\ \pause
\vspace{0.2cm}

When a pattern ``wins:'' \texttt{p1length.append(i)} or \texttt{p2length.append(i)}\\ \pause
\vspace{0.2cm}

At the end: Find the average of \texttt{p1length} and \texttt{p2length}.



\end{frame}

\plain{\href{https://repl.it/GbpB/1}{The Program (link)}}


\section{Results}

\begin{frame}{Results}
  The probability of Pattern 1 showing up first is: 66.67\%\\
  Average flips until we see Pattern 1 is: 5.67\\
  \vspace{0.5cm}

  The probability of Pattern 2 showing up first is: 33.33\%\\
  Average flips until we see Pattern 2 is: 4.67
\end{frame}

\plain{Why?}


\end{document}
